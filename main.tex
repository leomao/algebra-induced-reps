\documentclass[a4paper]{article}

%%%%%%%%%%%%%%%%%%page size%%%%%%%%%%%%%%%%%%
% \paperwidth=65cm
% \paperheight=160cm

%%%%%%%%%%%%%%%%%%%Package%%%%%%%%%%%%%%%%%%%
\usepackage[margin=3cm]{geometry}
\usepackage{mathtools,amsthm,amssymb}
\usepackage{centernot}
\usepackage{yhmath}
\usepackage{graphicx}
\usepackage{fontspec}
\usepackage{titlesec}
\usepackage{titling}
\usepackage{fancyhdr}
\usepackage{tabularx}
\usepackage[square, comma, numbers, super, sort&compress]{natbib}
\usepackage[unicode, pdfborder={0 0 0}, bookmarksdepth=-1]{hyperref}
\usepackage[usenames, dvipsnames]{color}
\usepackage[shortlabels, inline]{enumitem}
\usepackage{xpatch}

%\usepackage{tabto}     
%\usepackage{soul}      
%\usepackage{ulem}      
%\usepackage{wrapfig}   
%\usepackage{floatflt}  
\usepackage{float}     
\usepackage{caption}   
\usepackage{subcaption}
%\usepackage{setspace}  
\usepackage{mdframed}  
%\usepackage{multicol}  
%\usepackage[abbreviations]{siunitx}
%\usepackage{dsfont}   

%%%%%%%%%%%%%%%%%%%TikZ%%%%%%%%%%%%%%%%%%%%%%
\usepackage{tikz}
\usepackage{tikz-cd}
\usetikzlibrary{calc}
\usetikzlibrary{arrows}
\usetikzlibrary{shapes}
\usetikzlibrary{positioning}

\tikzstyle{every picture}+=[remember picture]

%%%%%%%%%%%%%%%%%font size%%%%%%%%%%%%%%%%%%%
%\def\normalsize{\fontsize{10}{15}\selectfont}
%\def\large{\fontsize{12}{18}\selectfont}
%\def\Large{\fontsize{14}{21}\selectfont}
%\def\LARGE{\fontsize{16}{24}\selectfont}
%\def\huge{\fontsize{18}{27}\selectfont}
%\def\Huge{\fontsize{20}{30}\selectfont}

%%%%%%%%%%%%%%%Theme Input%%%%%%%%%%%%%%%%%%%
%\input{themes/chapter/neat}
%\input{themes/env/problist}

%%%%%%%%%%%titlesec settings%%%%%%%%%%%%%%%%%
%\titleformat{\chapter}{\bf\Huge}
            %{\arabic{section}}{0em}{}
%\titleformat{\section}{\centering\Large}
            %{\arabic{section}}{0em}{}
%\titleformat{\subsection}{\large}
            %{\arabic{subsection}}{0em}{}
%\titleformat{\subsubsection}{\bf\normalsize}
            %{\arabic{subsubsection}}{0em}{}
%\titleformat{command}[shape]{format}{label}
            %{gutter}{before}[after]

%%%%%%%%%%%%variable settings%%%%%%%%%%%%%%%%
%\numberwithin{equation}{section}
%\setcounter{secnumdepth}{4}
%\setcounter{tocdepth}{1}
%\setcounter{section}{0}
%\graphicspath{{images/}}

%%%%%%%%%%%%%%%page settings%%%%%%%%%%%%%%%%%
\newcolumntype{C}[1]{>{\centering\arraybackslash}p{#1}}
\setlength{\headheight}{15pt}  % with titling
\setlength{\droptitle}{-2.5cm}
%\posttitle{\par\end{center}}  % distance between title and content
\parindent=0pt % indent size
\parskip=1ex    % line space
%\pagestyle{empty}  % empty: no page number
%\pagestyle{fancy}  % fancy: fancyhdr

% use with fancygdr
%\lhead{\leftmark}
%\chead{}
%\rhead{}
%\lfoot{}
%\cfoot{}
%\rfoot{\thepage}
%\renewcommand{\headrulewidth}{0.4pt}
%\renewcommand{\footrulewidth}{0.4pt}

%\fancypagestyle{firststyle}
%{
  %\fancyhf{}
  %\fancyfoot[C]{\footnotesize Page \thepage\ of \pageref{LastPage}}
  %\renewcommand{\headrule}{\rule{\textwidth}{\headrulewidth}}
%}

%%%%%%%%%%%%%%%renew command%%%%%%%%%%%%%%%%%
% \renewcommand{\contentsname}{Table of Content}
% \renewcommand{\refname}{Reference}
\renewcommand{\abstractname}{\LARGE Abstract}

%%%%%%%%symbol and function settings%%%%%%%%%
\DeclarePairedDelimiter{\abs}{\lvert}{\rvert}
\DeclarePairedDelimiter{\norm}{\lVert}{\rVert}
\DeclarePairedDelimiter{\inpd}{\langle}{\rangle} % inner product
\DeclarePairedDelimiter{\ceil}{\lceil}{\rceil}
\DeclarePairedDelimiter{\floor}{\lfloor}{\rfloor}
\DeclareMathOperator{\adj}{adj}
\DeclareMathOperator{\sech}{sech}
\DeclareMathOperator{\csch}{csch}
\DeclareMathOperator{\arcsec}{arcsec}
\DeclareMathOperator{\arccot}{arccot}
\DeclareMathOperator{\arccsc}{arccsc}
\DeclareMathOperator{\arccosh}{arccosh}
\DeclareMathOperator{\arcsinh}{arcsinh}
\DeclareMathOperator{\arctanh}{arctanh}
\DeclareMathOperator{\arcsech}{arcsech}
\DeclareMathOperator{\arccsch}{arccsch}
\DeclareMathOperator{\arccoth}{arccoth}
%%%% Math symbol %%%%
\newcommand*{\defeq}{\vcentcolon=}
\newcommand*{\Nb}{\mathbb{N}}
\newcommand*{\Zb}{\mathbb{Z}}
\newcommand*{\Qb}{\mathbb{Q}}
\newcommand*{\Rb}{\mathbb{R}}
\newcommand*{\Cb}{\mathbb{C}}
\newcommand*{\Hb}{\mathbb{H}}
\newcommand*{\Fb}{\mathbb{F}}
\newcommand*{\Fbx}{\mathbb{F}^\times}
\newcommand*{\Qbx}{\mathbb{Q}^\times}
\newcommand*{\Rbx}{\mathbb{R}^\times}
\newcommand*{\Cbx}{\mathbb{C}^\times}
\newcommand*{\Hbx}{\mathbb{H}^\times}
\newcommand*{\Gc}{\mathcal{G}}
\newcommand*{\Fc}{\mathcal{F}}
\newcommand*{\Cc}{\mathcal{C}}
\newcommand*{\Dc}{\mathcal{D}}
\newcommand*{\sT}{{\sf T}}
\newcommand*{\sI}{{\sf I}}

\newcommand*{\Mf}{\mathfrak{M}}
\newcommand*{\Grf}{\mathfrak{Gr}}

\DeclareMathOperator{\Res}{Res}
\DeclareMathOperator{\Ind}{Ind}

\newcommand*\GL[1]{\operatorname{GL}\mathopen{}\left({#1}\right)\mathclose{}}

\DeclareMathOperator{\Sym}{Sym}
\DeclareMathOperator{\Alt}{Alt}
\DeclareMathOperator{\diag}{diag}
\DeclareMathOperator{\sgn}{sgn}
\DeclareMathOperator{\lcm}{lcm}
\DeclareMathOperator{\Image}{Im}
\DeclareMathOperator{\Char}{char}
\DeclareMathOperator{\Fix}{Fix}
\DeclareMathOperator{\Inn}{Inn}
\DeclareMathOperator{\Aut}{Aut}
\DeclareMathOperator{\Isom}{Isom}
\DeclareMathOperator{\Tor}{Tor}
\DeclareMathOperator{\Exp}{Exp}
\DeclareMathOperator{\Syl}{Syl}

% multilinear
\DeclareMathOperator{\Hom}{Hom}
\DeclareMathOperator{\Lrad}{lrad}
\DeclareMathOperator{\Rrad}{rrad}
\DeclareMathOperator{\rank}{rank}
\DeclareMathOperator{\trace}{Tr}

% extensions
\DeclareMathOperator{\Stab}{Stab}
\DeclareMathOperator{\Der}{Der}
\DeclareMathOperator{\PDer}{PDer}
\DeclareMathOperator{\Ext}{Ext}
%

\newcommand*{\ob}{\overline}
\DeclareMathOperator{\ord}{ord}
\DeclarePairedDelimiter{\gen}{\langle}{\rangle} % generator
%\newcommand*\quot[2]{{^{\textstyle #1}\Big/_{\textstyle #2}}}
\newcommand*\quot[2]{{#1}/{#2}}
\newcommand*\bij{\lhook\joinrel\twoheadrightarrow}
\newcommand*\oneto{\hookrightarrow}
\newcommand*\onto{\twoheadrightarrow}
\newcommand*\isoto{\xrightarrow{\sim}}
\newcommand*\acts{\curvearrowright}
\newcommand*\revacts{\curvearrowleft}

% just to make sure it exists
\providecommand\given{}
% can be useful to refer to this outside \Set
\newcommand*\SetSymbol[1][]{%
  \nonscript\:#1\vert
  \allowbreak
  \nonscript\:
\mathopen{}}
\DeclarePairedDelimiterX\Set[1]\{\}{%
  \renewcommand\given{\SetSymbol[\delimsize]}
  \,#1\,
}

\DeclarePairedDelimiterX\Gen[1]{\langle}{\rangle}{%
  \renewcommand\given{\SetSymbol[\delimsize]}
  \,#1\,
}

% cycle group \cycle{1,2,3} => (1 2 3)
\ExplSyntaxOn
\NewDocumentCommand{\cycle}{ O{\;} m }
 {
  (
  \alec_cycle:nn { #1 } { #2 }
  )
 }

\seq_new:N \l_alec_cycle_seq
\cs_new_protected:Npn \alec_cycle:nn #1 #2
 {
  \seq_set_split:Nnn \l_alec_cycle_seq { , } { #2 }
  \seq_use:Nn \l_alec_cycle_seq { #1 }
 }
\ExplSyntaxOff

%%%%%%%%%%%%%%%%%%%%%%%%%%%%%%%%%%%%%%%%%%%%
%\renewcommand{\proofname}{\bf pf:}
\newtheoremstyle{mystyle}% custom style
  {6pt}{15pt}%      top and bottom margin
  {}%               content style
  {}%               indent
  {\bf}%            head style
  {.}%              after head
  {1em}%            distance between head and content
  {}%               Theorem head spec (can be left empty, meaning 'normal')

\theoremstyle{mystyle}
\newtheorem{theorem}{Thm}
\newtheorem{lemma}{Lemma}
\newtheorem{remark}{Remark}
\newtheorem{observation}{Obs}
\newtheorem{definition}{Def}
\newtheorem{example}{Example}
\newtheorem{exercise}{Ex}
\newtheorem{fact}{Fact}
\newtheorem{prop}{Prop}
\newtheorem{coro}{Coro}

%%%%%%%%%%%%%%Title information%%%%%%%%%%%%%%
\title{Induced Representations}
\author{Yao-Wen Mao}
\date{\today}

\begin{document}
\maketitle
% \thispagestyle{empty}
% \thispagestyle{fancy}
% \tableofcontents
%%%%%%%%%%%%%include file here%%%%%%%%%%%%%%%
\paragraph{Notations} \mbox{}
\begin{itemize}
  \item Throughout this handout, $G$ is a group and $H \le G$.
  \item All vector spaces are over $\Cb$.
  \item The set of left cosets $\Set{ gH \given g \in G }$ is denoted by $G/H$.
  \item $\abs{G/H}$ is the {\bf index} of $H$ in $G$, which is
    denoted by $[G:H]$.
  \item For a representation $\rho: G\to \GL{V}$, $\rho_g \in \GL{V}$ denotes
    $\rho(g)$.
  \item For a representation $\rho: G\to \GL{V}$, if a subspace $W$ of $V$
    is $G$-invariant, then
    \[
      \rho^W: G\to \GL{W}, \quad g \mapsto \rho(g)|_W
    \]
    is a subrepresentation of $\rho$.
\end{itemize}

\begin{definition}
  Given a representation $\rho: G \to \GL{V}$, we can restrict $\rho$ to $H$
  to obtain a representation $\rho|_H: H \to \GL{V}$. We also denote this
  representation by $\Res^G_H \rho : H\to \GL{V}$.
\end{definition}

\begin{definition}
  Given a representation $\rho: G\to \GL{V}$ and a $H$-invariant
  subspace $W \subset V$ w.r.t. $\rho|_H$. Let $x \in G/H$,
  $\forall g, g' \in x, \exists h \in H$ s.t. $g' = gh$. Then
  $\rho_g' W = \rho_g \rho_h W = \rho_g W$. That is, $\rho_g W$ only
  depends on the coset which $g$ belongs to.
  $\forall x \in G/H$, we denotes $\rho_g W$ by $W_x$ where $g \in x$.
\end{definition}

\begin{definition}
  A representation $\rho: G\to \GL{V}$ is said to be {\bf induced} by
  the representation $\theta: H\to \GL{W}$ if
  \[
    (\rho|_H)^W = \theta \quad \text{and} \quad V = \bigoplus_{x\in G/H} W_x.
  \]
  In particular, we have
  $\displaystyle \dim V = \sum_{x\in G/H} \dim W_x = [G:H] \cdot \dim W$.
\end{definition}

\begin{example}
  Consider $\displaystyle V = \sum_{g\in G} \Cb e_g$ with
  $\rho = \rho^\text{reg}$. Let $\theta: H\to \GL{W}, \quad h \mapsto \rho_h$
  be the restricted representation, where
  $\displaystyle W = \sum_{h\in H} \Cb e_h$. Obviously $\rho$ is
  induced by $\theta = \rho|_H$.
\end{example}

\begin{prop}
  If $\rho_1, \rho_2$ are induced by $\theta_1, \theta_2$ respectively, then
  $\rho_1 \oplus \rho_2$ is induced by $\theta_1 \oplus \theta_2$.

  \begin{proof}
    Let $\rho_1: G\to V_1, \rho_2: G\to V_2$ and
    $\theta_1: H\to W, \theta_2: H\to U$. We have
    \[
      V_1 = \bigoplus_{x\in G/H} W_x \qquad V_2 = \bigoplus_{x\in G/H} U_x
    \]
    So $V_1 \oplus V_2$ can be written as
    \[
      \left(\bigoplus_{x\in G/H} W_x\right) \oplus
      \left(\bigoplus_{x\in G/H} U_x\right)
      = \bigoplus_{x\in G/H} (W_x \oplus U_x)
      = \bigoplus_{x\in G/H} (W \oplus U)_x
    \]
    by the definition of the direct sum of two representations.
  \end{proof}
\end{prop}

\begin{prop}
  Let $\rho: G\to \GL{V}$ be induced by $\theta: H\to \GL{W}$. If $W_1$ is
  a $H$-invariant subspace of $W$, then the subspace
  $\displaystyle V_1 = \sum_{x \in G/H} (W_1)_x$ is $G$-invariant and
  $\rho^{V_1}$ is induced by $\theta^{W_1}$.
  \begin{proof}
    Notice that $V$ is the direct sum of $W_x$ and $(W_1)_x \subset W$, so
    $V_1$ must be the direct sum of $(W_1)_x$. And for any $g\in G$, we have
    \[
      \rho_g V_1 = \sum_{x\in G/H} \rho_g (W_1)_x
      = \sum_{x\in G/H} (W_1)_{g\cdot x}
      = \sum_{x'\in G/H} (W_1)_{x'} = V_1
    \]
    where $g\cdot x$ is the left multiplication action of $G$ on $G/H$.
    So $V_1$ is $G$-invariant.
  \end{proof}
\end{prop}

Now, we try to reformulate all the things in terms of module:

\begin{definition}
  Given $\rho: G\to \GL{V}$, it defines a $G$-action on $V$ by
  $g\cdot v = \rho_g(v)$. We can regard $V$ as a $\Cb[G]$-module.
  $\forall v \in V$,
  \[
    \left(\sum_{g\in G} r_g g\right) \cdot v
    =\sum_{g\in G} r_g (g\cdot v)
  \]
\end{definition}

\begin{definition}
  Let $W$ be a $\Cb[H]$-module. We define a $\Cb[G]$-module $\Ind^G_H W$ as
  \[
    \Ind^G_H W = \Cb[G] \otimes_{\Cb[H]} W
  \]
\end{definition}

\begin{observation}
  Let $\rho: G \to \GL{V}$ and $W$ be a $H$-invariant subspace of $V$.
  Define a map
  \[
    f: \Cb[G] \times W \to V, \quad
    \left(\sum_{g\in G} r_g g\right) \otimes w \mapsto
    \sum_{g\in G} r_g \rho_g(w)
  \]
  This is $\Cb[H]$-bilinear with $h \cdot w = \rho_h(w), \quad \forall h\in H,
  w\in W$.
  By the universal property, $\exists!$ $\Cb[H]$-module homomorphism
  $\varphi: \Ind^G_H W \to V$ s.t.
  \[
    \begin{tikzcd}
      \Cb[G] \times W \arrow[r] \arrow[dr, "f"']
      & \Ind^G_H W \arrow[d, "\varphi"] \\
      & V
    \end{tikzcd}
  \]
  Actually, we can extend $\varphi$ to a $\Cb[G]$-module homomorphism
  since $\Ind^G_H W$ can be regarded as a $\Cb[G]$-module and $V$ is also a
  $\Cb[G]$-module.
\end{observation}

\begin{observation}
  $\forall gH \in G/H$, let $gH \otimes_{\Cb[H]} W$ denotes the set
  \[
    \Gen*{ gh \otimes w \given \forall h \in H, w \in W}_{\Cb 1_H}
  \]
  It is easy to find that
  \[
    \Ind^G_H W = \Cb[G] \otimes_{\Cb[H]} W
    \cong \bigoplus_{x \in G/H} x \otimes_{\Cb[H]} W
  \]
\end{observation}

\begin{theorem}[Existence and Uniqueness of the Induced Representation]

\end{theorem}

% the character of the induced reps

\begin{theorem}[Mackey's theorem]
\end{theorem}

%%%%%%%%%%%%%%%%%%%%%%%%%%%%%%%%%%%%%%%%%%%%%
% \bibliographystyle{plain}
% \bibliography{journal.bib}
% \begin{thebibliography}{99}
% \bibitem[1]{ex}\url{http://www.example.com/}
% \end{thebibliography}
\end{document}
