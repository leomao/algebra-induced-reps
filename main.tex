\documentclass[a4paper]{article}

%%%%%%%%%%%%%%%%%%page size%%%%%%%%%%%%%%%%%%
% \paperwidth=65cm
% \paperheight=160cm

%%%%%%%%%%%%%%%%%%%Package%%%%%%%%%%%%%%%%%%%
\usepackage[margin=3cm]{geometry}
\usepackage{mathtools,amsthm,amssymb}
\usepackage{centernot}
\usepackage{yhmath}
\usepackage{graphicx}
\usepackage{fontspec}
\usepackage{titlesec}
\usepackage{titling}
\usepackage{fancyhdr}
\usepackage{tabularx}
\usepackage[square, comma, numbers, super, sort&compress]{natbib}
\usepackage[unicode, pdfborder={0 0 0}, bookmarksdepth=-1]{hyperref}
\usepackage[usenames, dvipsnames]{color}
\usepackage[shortlabels, inline]{enumitem}
\usepackage{xpatch}

%\usepackage{tabto}     
%\usepackage{soul}      
%\usepackage{ulem}      
%\usepackage{wrapfig}   
%\usepackage{floatflt}  
\usepackage{float}     
\usepackage{caption}   
\usepackage{subcaption}
%\usepackage{setspace}  
\usepackage{mdframed}  
%\usepackage{multicol}  
%\usepackage[abbreviations]{siunitx}
%\usepackage{dsfont}   

%%%%%%%%%%%%%%%%%%%TikZ%%%%%%%%%%%%%%%%%%%%%%
\usepackage{tikz}
\usepackage{tikz-cd}
\usetikzlibrary{calc}
\usetikzlibrary{arrows}
\usetikzlibrary{shapes}
\usetikzlibrary{positioning}

\tikzstyle{every picture}+=[remember picture]

%%%%%%%%%%%%%%%%%font size%%%%%%%%%%%%%%%%%%%
%\def\normalsize{\fontsize{10}{15}\selectfont}
%\def\large{\fontsize{12}{18}\selectfont}
%\def\Large{\fontsize{14}{21}\selectfont}
%\def\LARGE{\fontsize{16}{24}\selectfont}
%\def\huge{\fontsize{18}{27}\selectfont}
%\def\Huge{\fontsize{20}{30}\selectfont}

%%%%%%%%%%%%%%%Theme Input%%%%%%%%%%%%%%%%%%%
%\input{themes/chapter/neat}
%\input{themes/env/problist}

%%%%%%%%%%%titlesec settings%%%%%%%%%%%%%%%%%
%\titleformat{\chapter}{\bf\Huge}
            %{\arabic{section}}{0em}{}
%\titleformat{\section}{\centering\Large}
            %{\arabic{section}}{0em}{}
%\titleformat{\subsection}{\large}
            %{\arabic{subsection}}{0em}{}
%\titleformat{\subsubsection}{\bf\normalsize}
            %{\arabic{subsubsection}}{0em}{}
%\titleformat{command}[shape]{format}{label}
            %{gutter}{before}[after]

%%%%%%%%%%%%variable settings%%%%%%%%%%%%%%%%
%\numberwithin{equation}{section}
%\setcounter{secnumdepth}{4}
%\setcounter{tocdepth}{1}
%\setcounter{section}{0}
%\graphicspath{{images/}}

%%%%%%%%%%%%%%%page settings%%%%%%%%%%%%%%%%%
\newcolumntype{C}[1]{>{\centering\arraybackslash}p{#1}}
\setlength{\headheight}{15pt}  % with titling
\setlength{\droptitle}{-2.5cm}
%\posttitle{\par\end{center}}  % distance between title and content
\parindent=0pt % indent size
\parskip=1ex    % line space
%\pagestyle{empty}  % empty: no page number
%\pagestyle{fancy}  % fancy: fancyhdr

% use with fancygdr
%\lhead{\leftmark}
%\chead{}
%\rhead{}
%\lfoot{}
%\cfoot{}
%\rfoot{\thepage}
%\renewcommand{\headrulewidth}{0.4pt}
%\renewcommand{\footrulewidth}{0.4pt}

%\fancypagestyle{firststyle}
%{
  %\fancyhf{}
  %\fancyfoot[C]{\footnotesize Page \thepage\ of \pageref{LastPage}}
  %\renewcommand{\headrule}{\rule{\textwidth}{\headrulewidth}}
%}

%%%%%%%%%%%%%%%renew command%%%%%%%%%%%%%%%%%
% \renewcommand{\contentsname}{Table of Content}
% \renewcommand{\refname}{Reference}
\renewcommand{\abstractname}{\LARGE Abstract}

%%%%%%%%symbol and function settings%%%%%%%%%
\DeclarePairedDelimiter{\abs}{\lvert}{\rvert}
\DeclarePairedDelimiter{\norm}{\lVert}{\rVert}
\DeclarePairedDelimiter{\inpd}{\langle}{\rangle} % inner product
\DeclarePairedDelimiter{\ceil}{\lceil}{\rceil}
\DeclarePairedDelimiter{\floor}{\lfloor}{\rfloor}
\DeclareMathOperator{\adj}{adj}
\DeclareMathOperator{\sech}{sech}
\DeclareMathOperator{\csch}{csch}
\DeclareMathOperator{\arcsec}{arcsec}
\DeclareMathOperator{\arccot}{arccot}
\DeclareMathOperator{\arccsc}{arccsc}
\DeclareMathOperator{\arccosh}{arccosh}
\DeclareMathOperator{\arcsinh}{arcsinh}
\DeclareMathOperator{\arctanh}{arctanh}
\DeclareMathOperator{\arcsech}{arcsech}
\DeclareMathOperator{\arccsch}{arccsch}
\DeclareMathOperator{\arccoth}{arccoth}
%%%% Math symbol %%%%
\newcommand*{\defeq}{\vcentcolon=}
\newcommand*{\Nb}{\mathbb{N}}
\newcommand*{\Zb}{\mathbb{Z}}
\newcommand*{\Qb}{\mathbb{Q}}
\newcommand*{\Rb}{\mathbb{R}}
\newcommand*{\Cb}{\mathbb{C}}
\newcommand*{\Hb}{\mathbb{H}}
\newcommand*{\Fb}{\mathbb{F}}
\newcommand*{\Fbx}{\mathbb{F}^\times}
\newcommand*{\Qbx}{\mathbb{Q}^\times}
\newcommand*{\Rbx}{\mathbb{R}^\times}
\newcommand*{\Cbx}{\mathbb{C}^\times}
\newcommand*{\Hbx}{\mathbb{H}^\times}
\newcommand*{\Gc}{\mathcal{G}}
\newcommand*{\Fc}{\mathcal{F}}
\newcommand*{\Cc}{\mathcal{C}}
\newcommand*{\Dc}{\mathcal{D}}
\newcommand*{\sT}{{\sf T}}
\newcommand*{\sI}{{\sf I}}

\newcommand*{\Mf}{\mathfrak{M}}
\newcommand*{\Grf}{\mathfrak{Gr}}

\DeclareMathOperator{\Res}{Res}
\DeclareMathOperator{\Ind}{Ind}

\newcommand*\GL[1]{\operatorname{GL}\mathopen{}\left({#1}\right)\mathclose{}}

\DeclareMathOperator{\Sym}{Sym}
\DeclareMathOperator{\Alt}{Alt}
\DeclareMathOperator{\diag}{diag}
\DeclareMathOperator{\sgn}{sgn}
\DeclareMathOperator{\lcm}{lcm}
\DeclareMathOperator{\Image}{Im}
\DeclareMathOperator{\Char}{char}
\DeclareMathOperator{\Fix}{Fix}
\DeclareMathOperator{\Inn}{Inn}
\DeclareMathOperator{\Aut}{Aut}
\DeclareMathOperator{\Isom}{Isom}
\DeclareMathOperator{\Tor}{Tor}
\DeclareMathOperator{\Exp}{Exp}
\DeclareMathOperator{\Syl}{Syl}

% multilinear
\DeclareMathOperator{\Hom}{Hom}
\DeclareMathOperator{\Lrad}{lrad}
\DeclareMathOperator{\Rrad}{rrad}
\DeclareMathOperator{\rank}{rank}
\DeclareMathOperator{\trace}{Tr}

% extensions
\DeclareMathOperator{\Stab}{Stab}
\DeclareMathOperator{\Der}{Der}
\DeclareMathOperator{\PDer}{PDer}
\DeclareMathOperator{\Ext}{Ext}
%

\newcommand*{\ob}{\overline}
\DeclareMathOperator{\ord}{ord}
\DeclarePairedDelimiter{\gen}{\langle}{\rangle} % generator
%\newcommand*\quot[2]{{^{\textstyle #1}\Big/_{\textstyle #2}}}
\newcommand*\quot[2]{{#1}/{#2}}
\newcommand*\bij{\lhook\joinrel\twoheadrightarrow}
\newcommand*\oneto{\hookrightarrow}
\newcommand*\onto{\twoheadrightarrow}
\newcommand*\isoto{\xrightarrow{\sim}}
\newcommand*\acts{\curvearrowright}
\newcommand*\revacts{\curvearrowleft}

\newcommand*\bsl{\backslash}

% just to make sure it exists
\providecommand\given{}
% can be useful to refer to this outside \Set
\newcommand*\SetSymbol[1][]{%
  \nonscript\:#1\vert
  \allowbreak
  \nonscript\:
\mathopen{}}
\DeclarePairedDelimiterX\Set[1]\{\}{%
  \renewcommand\given{\SetSymbol[\delimsize]}
  \,#1\,
}

\DeclarePairedDelimiterX\Gen[1]{\langle}{\rangle}{%
  \renewcommand\given{\SetSymbol[\delimsize]}
  \,#1\,
}

% cycle group \cycle{1,2,3} => (1 2 3)
\ExplSyntaxOn
\NewDocumentCommand{\cycle}{ O{\;} m }
 {
  (
  \alec_cycle:nn { #1 } { #2 }
  )
 }

\seq_new:N \l_alec_cycle_seq
\cs_new_protected:Npn \alec_cycle:nn #1 #2
 {
  \seq_set_split:Nnn \l_alec_cycle_seq { , } { #2 }
  \seq_use:Nn \l_alec_cycle_seq { #1 }
 }
\ExplSyntaxOff

%%%%%%%%%%%%%%%%%%%%%%%%%%%%%%%%%%%%%%%%%%%%
%\renewcommand{\proofname}{\bf pf:}
\newtheoremstyle{mystyle}% custom style
  {6pt}{15pt}%      top and bottom margin
  {}%               content style
  {}%               indent
  {\bf}%            head style
  {.}%              after head
  {1em}%            distance between head and content
  {}%               Theorem head spec (can be left empty, meaning 'normal')

\theoremstyle{mystyle}
\newtheorem{theorem}{Thm}
\newtheorem{lemma}{Lemma}
\newtheorem{remark}{Remark}
\newtheorem{observation}{Obs}
\newtheorem{definition}{Def}
\newtheorem{example}{Example}
\newtheorem{exercise}{Ex}
\newtheorem{fact}{Fact}
\newtheorem{prop}{Prop}
\newtheorem{coro}{Coro}

%%%%%%%%%%%%%%Title information%%%%%%%%%%%%%%
\title{Induced Representations}
\author{Yao-Wen Mao}
\date{1/4, 2017}

\begin{document}
\maketitle
% \thispagestyle{empty}
% \thispagestyle{fancy}
% \tableofcontents
%%%%%%%%%%%%%include file here%%%%%%%%%%%%%%%
\paragraph{Notations} \mbox{}
\begin{itemize}
  \item Throughout this handout, $G$ is a group and $H \le G$.
  \item All vector spaces are over $\Cb$.
  \item The set of left cosets $\Set{ gH \given g \in G }$ is denoted by $G/H$.
  \item $\abs{G/H}$ is the {\bf index} of $H$ in $G$, which is
    denoted by $[G:H]$.
  \item For a representation $\rho: G\to \GL{V}$, $\rho_g \in \GL{V}$ denotes
    $\rho(g)$.
  \item For a representation $\rho: G\to \GL{V}$, if a subspace $W$ of $V$
    is $G$-invariant, then
    \[
      \rho^W: G\to \GL{W}, \quad g \mapsto \rho(g)|_W
    \]
    is a subrepresentation of $\rho$.
  \item Picking an element from each coset in $G/H$ forms a 
    {\bf set of representatives} $S$ of $G/H$.
    We write $s\in S$ as $s\in G/H$ for convenience.
\end{itemize}

\paragraph{Motivation} \mbox{}
\begin{definition}
  Given a representation $\rho: G \to \GL{V}$, we can restrict $\rho$ to $H$
  to obtain a representation $\rho|_H: H \to \GL{V}$. We also denote this
  representation by $\Res^G_H V$ or $\Res^G_H \rho : H\to \GL{V}$.
\end{definition}

\begin{definition}
  Given a representation $\rho: G\to \GL{V}$ and a $H$-invariant
  subspace $W \subset V$ w.r.t. $\rho|_H$. Let $s\in G/H$, $\forall g \in sH$,
  $\exists! h \in H$ s.t. $g = sh$.
  Then $\rho_g W = \rho_s \rho_h W = \rho_s W$. That is, $\rho_g W$ only
  depends on the coset which $g$ belongs to. $\forall s \in G/H$,
  we denotes $\rho_s W$ by $sW$.
\end{definition}

\begin{definition}
  A representation $\rho: G\to \GL{V}$ is said to be {\bf induced} by
  the representation $\theta: H\to \GL{W}$ if
  \[
    (\rho|_H)^W = \theta \quad \text{and} \quad V = \bigoplus_{s\in G/H} sW.
  \]
  In particular, we have
  $\displaystyle \dim V = \sum_{s\in G/H} \dim sW = [G:H] \cdot \dim W$.
\end{definition}

\begin{example}
  Consider $\displaystyle V = \sum_{g\in G} \Cb e_g$ with
  $\rho = \rho^\text{reg}$. Let $\theta: H\to \GL{W}, \quad h \mapsto \rho_h$
  be the restricted representation, where
  $\displaystyle W = \sum_{h\in H} \Cb e_h$. Obviously $\rho$ is
  induced by $\theta = (\rho|_H)^W$.
\end{example}

\begin{prop}
  If $\rho_1, \rho_2$ are induced by $\theta_1, \theta_2$ respectively, then
  $\rho_1 \oplus \rho_2$ is induced by $\theta_1 \oplus \theta_2$.

  \begin{proof}
    Let $\rho_1: G\to V_1, \rho_2: G\to V_2$ are induced by
    $\theta_1: H\to W, \theta_2: H\to U$ respectively. We have
    \[
      V_1 = \bigoplus_{s\in G/H} sW \qquad V_2 = \bigoplus_{s\in G/H} sU
    \]
    So $V_1 \oplus V_2$ can be written as
    \[
      \left(\bigoplus_{s\in G/H} sW\right) \oplus
      \left(\bigoplus_{s\in G/H} sU\right)
      = \bigoplus_{s\in G/H} (sW \oplus sU)
      = \bigoplus_{s\in G/H} s(W \oplus U)
    \]
    by the definition of the direct sum of two representations.
  \end{proof}
\end{prop}

\begin{prop}
  Let $\rho: G\to \GL{V}$ be induced by $\theta: H\to \GL{W}$. If $W_1$ is
  a $H$-invariant subspace of $W$, then the subspace
  $\displaystyle V_1 = \sum_{s \in G/H} sW_1$ is $G$-invariant and
  $\rho^{V_1}$ is induced by $\theta^{W_1}$.
  \begin{proof}
    Notice that $V$ is the direct sum of $sW$ and $sW_1 \subset W$, so
    $V_1$ must be the direct sum of $sW_1$. And for any $g\in G$, we have
    \[
      \rho_g V_1 = \sum_{s\in G/H} \rho_g sW_1
      = \sum_{s\in G/H} (gs)W_1
      = \sum_{s'\in G/H} s'W_1 = V_1
    \]
    where $\Set{ gs \given s\in G/H }$ is another set of representatives of
    $G/H$. So $V_1$ is $G$-invariant.
  \end{proof}
\end{prop}

\paragraph{Induced Module}
Now, we try to reformulate all the things in terms of module:

\begin{definition}
  Given $\rho: G\to \GL{V}$, it defines a $G$-action on $V$ by
  $g\cdot v = \rho_g(v)$. We can regard $V$ as a $\Cb[G]$-module.
  $\forall v \in V$,
  \[
    \left(\sum_{g\in G} c_g g\right) \cdot v
    =\sum_{g\in G} c_g (g\cdot v)
  \]
\end{definition}

\begin{definition}[Universal Property]
  Let $W$ be a $\Cb[H]$-module, $U$ be a $\Cb[G]$-module.
  An induced representation of $W$ is a $\Cb[G]$-module $\Ind^G_H W$ and the
  map $i: W \to \Ind^G_H W$ satisfying the universal property:
  
  For any $\Cb[H]$-module homomorphism $f: W\to U$, $\exists!$ $\Cb[G]$-module
  homomorphism $\hat{f}$ s.t.
  \[
    \begin{tikzcd}
      W \arrow[r, "i"] \arrow[dr, "f"'] & \Ind^G_H W \arrow[d, "\hat{f}"] \\
                                        & U
    \end{tikzcd}
  \]
\end{definition}

\begin{observation}
  If $\rho: G\to \GL{V}$ is induced by $\theta: H\to \GL{W}$, i.e.
  \[
    V = \bigoplus_{s \in G/H} sW,
  \]
  then $V$ satisfies the universal property with the inclusion map:
  $i: W\to V, w \mapsto w$.
  \begin{proof}
    For each $s \in G/H$, consider the map
    \[
      f_s: sW \to U, \quad w_s \mapsto sf(s^{-1}w_s)
    \]
    For another $g \in sH$, $\exists h \in H$ s.t. $g = sh$, then
    $gf(g^{-1}w_s) = shf(h^{-1}s^{-1}w_s) = sf(s^{-1}w_s)$ since
    $f$ is a $\Cb[H]$-module homomorphism.
    $\implies$ This map doesn't depend on the choice of representatives.

    Since each $f_s$ is a $\Cb$-module homomorphism,
    by the universal property of the direct sum, $\exists!$ $\Cb$-module
    homomorphism $\hat{f}$ s.t.
    \[
      \begin{tikzcd}
        sW \arrow[r, "i"] \arrow[dr, "f_s"'] & V \arrow[d, "\hat{f}"] \\
                                              & U
      \end{tikzcd}
    \]
    Now we verify that $\hat{f}$ is $G$-equivariant.
    Since $\hat{f}$ is $\Cb$-module homomorphism, we can just check
    $\forall s \in G/H, g\in G, w_s \in sW, g\hat{f}(w_s) = \hat{f}(gw_s)$.
    By the definition, $\exists w_{gs} \in W_{gs}$ s.t. $w_{gs} = gw_s$.
    So
    \[
      \hat{f}(gw_s) = \hat{f}(w_{gs})
                    = gsf(s^{-1}g^{-1}w_{gs})
                    = gsf(s^{-1}w_s)
                    = g\hat{f}(w_s)
    \]
    We conclude that $\hat{f}$ is a $\Cb[G]$-module homomorphism.
    
    Now for any $w_s \in W_s$, $\exists w \in W$ s.t. $w_s = gw$,
    \[
      \hat{f}(w_s) = \hat{f}(sw) = s\hat{f}(w) = sf(w)
    \]
    This doesn't depend on the choice of representatives. So $\hat{f}$ is
    determined by $f$, hence $\hat{f}$ is unique.
  \end{proof}
\end{observation}

\begin{theorem}[Existence and Uniqueness of the Induced Representation]
  Given a representation $W$ of $H$, the induced representation
  $\Ind^G_H W$ of $G$ exists and is unique up to isomorphism.
  \begin{proof}
    \begin{description}
      \item[Existence:]
        We just prove that the $\Cb[G]$-module $\Cb[G] \otimes_{\Cb[H]} W$
        satisfies the universal property.
        The inclusion map is $i: W \to \Cb[G]\otimes_{\Cb[H]} W,
        \quad w \mapsto (1\otimes w)$.
        
        First we define a $\Cb[H]$-bilinear map
        \[
          \tilde{f}: \Cb[G] \times W \to U, \quad
          \left(\sum_{g\in G} c_g g, w\right) \mapsto
          \sum_{g\in G} c_g gf(w).
        \]
        By the universal property of the tensor product, $\exists!$
        $\Cb[H]$-module homomorphism $\hat{f}$ s.t.
        \[
          \begin{tikzcd}
            \Cb[G] \times W \arrow[r, "\hat{i}"] \arrow[dr, "\tilde{f}"']
            & \Cb[G]\otimes_{\Cb[H]}W \arrow[d, "\hat{f}"] \\
            & U
          \end{tikzcd}
        \]
        Verifying that $\hat{f}$ is $G$-equivariant is easy.
        So $\hat{f}$ is a $\Cb[G]$-module homomorphism.
        In particular, we have $\hat{f}(1 \otimes w) = \tilde{f}(1, w) = f(w)$.
        
        To prove $\hat{f}$ is unique, It is sufficient to show that for all
        elements in the generating set of $\Cb[G] \otimes_{\Cb[H]} W$,
        the value of $\hat{f}$ can be determined by $f$.
        Note that $\hat{f}$ is a $\Cb[G]$-module homomorphism. So
        $\forall c \in \Cb[G], w\in W$, $\hat{f}(c \otimes w)
        = \hat{f}(c (1\otimes w)) = c\hat{f}(1\otimes w) = cf(w)$.
        So $\hat{f}$ is unique.
      \item[Uniqueness:] Just like how we prove the uniqueness for
        other universal properties.
        \qedhere
    \end{description}
  \end{proof}
\end{theorem}

\begin{definition}
  By the theorem above, we can define $\Ind^G_H W = \Cb[G] \otimes_{\Cb[H]} W$.
\end{definition}

\begin{remark}
  Induction is transitive: If $H \le K \le G$, then
  \[
    \Ind^G_K (\Ind^K_H W) \cong \Ind^G_H W
  \]
\end{remark}

\begin{prop}
  Let $V$ be a $\Cb[G]$-module which is a direct sum
  $V = \bigoplus_{i \in A} W_i$ of vector subspaces {\it permuted transitively}
  by $G$, $A$ is a index set. Let $i_0 \in A, W = W_{i_0}$ and
  $H$ be the stabilizer of $W$ in $G$.
  Then $W$ is a $\Cb[H]$-module and $V$ is induced by $W$.
\end{prop}

% the character of the induced reps

\paragraph{Character of the Induced Representation}
Now, suppose $\rho: G\to \GL{V}$ is induced by $\theta: H\to \GL{W}$.
Since $\theta$ determines $\rho$ up to isomorphism, we ought to compute
$\chi_\rho$ from $\chi_\theta$.

\begin{definition}
  Let $f \in \Cc(G)$ be a class function on $G$. Then
  $\Res^G_H f = f|_H$ is a class function on $H$, i.e.
  $\Res^G_H f \in \Cc(H)$.
\end{definition}

\begin{definition}
  Let $f \in \Cc(H)$ be a class function on $H$, consider the function
  $f'$ on $G$ defined as
  \[
    f'(u) = \frac{1}{\abs{H}} \sum_{\substack{t\in G\\ t^{-1}ut \in H}}
    f(t^{-1}ut) \quad \forall u\in G
  \]
  We say that $f'$ is induced by $f$ and denoted by $\Ind^G_H f$.
  Note that $f' \in \Cc(G)$ since
  \[
    f'(v^{-1}uv) = \frac{1}{\abs{H}} \sum_{\substack{t\in G\\ t^{-1}(v^{-1}uv)t \in H}}
    f(t^{-1}(v^{-1}uv)t)
    = \frac{1}{\abs{H}} \sum_{\substack{t'\in G\\ t'^{-1}ut' \in H}}
    f(t'^{-1}ut') = f'(u) \quad \forall u, v\in G
  \]
\end{definition}

\begin{prop}
  \label{prob:linear}
  $\Res^G_H: \Cc(G) \to \Cc(H)$ and $\Ind^G_H: \Cc(H) \to \Cc(G)$ are both
  $\Cb$-module homomorphism.
\end{prop}

\begin{theorem}[Frobenius Reciprocity]
  \label{thm:frob}
  Let $\theta: H\to \GL{W}$ and $\rho: G\to \GL{V}$ be representations of
  $H, G$ respectively. Then there is a unique isomorphism
  \[
    \varphi: \Hom_H(W, \Res^G_H V) \isoto \Hom_G(\Ind^G_H W, V), \quad
    \alpha \mapsto \beta
  \]
  such that $\alpha(w) = \beta(w), \quad \forall w \in W$.
  $(\Ind^G_H, \Res^G_H)$ are adjoint functors.
  \begin{proof}
    Given $\alpha \in \Hom_H(W, \Res^G_H V)$, define
    $\beta \in \Hom_G(\Ind^G_H W, V)$ by the universal property.
    This map is $\Cb$-module homomorphism obviously.

    Now, given $\beta \in \Hom_G(\Ind^G_H W, V)$, define
    $\alpha \in \Hom_H(W, \Res^G_H V)$ by
    \[
      \alpha(w) = \beta(1\otimes w)
    \]
    Then clearly $\alpha = \beta \circ i$, where $i: W\to \Ind^G_H W$ is the
    inclusion map.

    By the universal property, $\beta$ is unique, so $\varphi(\alpha) = \beta$.
    This mapping is an inverse of $\varphi$.
    
    $\implies$ $\varphi$ is an isomorphism.
  \end{proof}
\end{theorem}

\begin{definition}
  Let $V_1, V_2$ be two $\Cb[G]$-modules, we define
  $\inpd{V_1, V_2}_G = \dim \Hom_G(V_1, V_2)$.
\end{definition}

\begin{prop}
  \label{prop:dimop}
  Let $W_1, W_2, V$ be $\Cb[G]$-modules. Then
  \begin{align*}
    \Hom_G(W_1 \oplus W_2, V) &= \Hom_G(W_1, V) \oplus \Hom_G(W_2, V) \\
    \Hom_G(V, W_1 \oplus W_2) &= \Hom_G(V, W_1) \oplus \Hom_G(V, W_2)
  \end{align*}
  \begin{proof}
    Just like how we prove Thm \ref{thm:frob},
    this is easy to show by the universal property of the direct sum
    and the direct product.
    (In this case, the direct sum can be regard as the direct product.)
  \end{proof}
\end{prop}

\begin{lemma}
  \label{lemma:inpd}
  If $\phi_1, \phi_2$ are the characters of $V_1, V_2$, we have
  $\inpd{\phi_1, \phi_2}_G = \inpd{V_1, V_2}_G$.
  \begin{proof}
    By Prop \ref{prop:dimop}, we only need to check when $\phi_1, \phi_2$
    are irreducible characters.
    Then this is just the result of Schur's lemma.
  \end{proof}
\end{lemma}

\begin{coro}
  \label{coro:chi}
  Let $\rho: G\to \GL{V}$ be induced by $\theta:H\to \GL{W}$ and
  $R$ be a representatives of $G/H$. Then $\chi_\rho = \Ind^G_H \chi_\theta$.
  \begin{proof}
    By the uniqueness of the induced representation, $V = \Ind^G_H W$.
    Let $\chi$ be an irreducible character of $G$ with
    $\sigma: G\to \GL{U}$. We have
    \[
      \inpd{\chi_\rho, \chi}_G = \inpd{\Ind^G_H W, U}_G
      = \inpd{W, \Res^G_H U}_H = \inpd{\chi_\theta, \Res^G_H \chi}_H
      = \inpd{\Ind^G_H \chi_\theta, \chi}_G
    \]
    So $\chi_\rho = \Ind^G_H \chi_\theta$.
  \end{proof}
\end{coro}

\begin{coro}[Frobenius Formula]
  Let $\psi \in \Cc(H)$, $\phi \in \Cc(G)$. Then
  \[
    \inpd{\psi,\Res^G_H \phi}_H = \inpd{\Ind^G_H \psi,\phi}_G.
  \]
  \begin{proof}
    Since each class function is a linear combination of irreducible
    characters, this can be obtained directly by Prob \ref{prob:linear}
    , Lemma \ref{lemma:inpd}, Thm \ref{thm:frob} and Coro \ref{coro:chi}.
  \end{proof}
\end{coro}

\paragraph{Restrict the Induced Representation}
Now we are interesting in the restricted representation of the induced
representation. Let $K \le G$. $\rho: H\to \GL{W}$ is a representation
of $H$. $\Ind^G_H W$ is the induced representation.

\begin{definition}
  The set of double cosets of $K, H$ is $K \bsl G / H \defeq
  \Set{ KgH \given g \in G}$.
  Let $S$ be a set of representatives of $K\bsl G/H$, i.e.
  $G$ is the disjoint union of $KsH$ for $s\in S$.
  For convenience, we write $s \in K\bsl G/H$ to denote $s\in S$.
\end{definition}

\begin{definition}
  For $s \in K\bsl G/H$, let $H_s = sHs^{-1} \cap K$,
  which is a subgroup of $K$.
  ($H_s$ doens't depend on the choice of the representative)
\end{definition}

\begin{definition}
  Define $\rho^s: H_s \to \GL{W}, \quad x \mapsto \rho(s^{-1}xs)$.
  Then this is a representation of $H_s$, denoted as $W_s$.
  Since $H_s \le K$, $\Ind^K_{H_s} W_s$ is defined.
\end{definition}

\begin{observation}
  We know that $\Ind^G_H W \cong \bigoplus_{x\in G/H} xW$. Let $V = \Ind^G_H W$.
  Given $s \in K\bsl G/H$, consider $x \in KsH$. We can write $x = ksh$ for
  some $k\in K, h \in H$, then $xW = kshW = ksW$. So we can just consider
  $x \in Ks$.
  Now for $k_1s, k_2s\in Ks$, if $k_1sW = k_2sW$, then $\exists h\in H$ s.t.
  $k_1s = k_2sh \implies k_2^{-1}k_1 = shs^{-1} \in H_s \implies
  k_1 \in k_2H_s$.

  We found that the subspaces generated by $\Set{ xW \given x\in KsH}$ is
  same as $\Set{ xsW \given x \in K/H_s}$ and any two elements in this
  set are disjoint subspaces.

  We define $\displaystyle V_s = \bigoplus_{x\in K/H_s} xsW$ which is a
  $K$-invariant subspace of $V$. Also, it is not difficult to show that for
  two distinct $s_1, s_2 \in K\bsl G/H$, $V_{s_1}, V_{s_2}$ are disjoint.
  So $\displaystyle V = \bigoplus_{s\in K\bsl G/H} V_s$.
\end{observation}

\begin{theorem}[Mackey's Theorem]
  \label{thm:mackey}
  $\displaystyle \Res^G_K \Ind^G_H W = \bigoplus_{s\in K\bsl G/H} \Ind^K_{H_s} W_s$.
  \begin{proof}
    Let $V = \Ind^G_H W$. By the observation above, we have
    \[
      V_s = \bigoplus_{x\in K/H_s} xsW = \Ind^K_{H_s} sW
    \]
    We only need to show that $\varphi: W_s \to sW, \quad w \mapsto sw$
    is a $\Cb[H_s]$-module isomorphism.
    This is obvious: $\forall t\in H_s, \varphi(tw) = \varphi(\rho(s^{-1}ts)(w))
    = \varphi(s^{-1}tsw) = tsw = t\varphi(w)$.
  \end{proof}
\end{theorem}

\begin{definition}
  Two representations $V_1, V_2$ of a group $G$ are said to be {\bf disjoint}
  if they have no common irreducible subrepresentation, i.e.
  $\inpd{V_1, V_2}_G = 0$.
\end{definition}

\begin{theorem}[Mackey's Criterion]
  Let $K = H$. The induced representation $V = \Ind^G_H W$ is irreducible
  if and only if
  \begin{enumerate}[(a)]
    \item $W$ is irreducible, and
    \item for each $s\in G - H$, the two representations $W_s$
      and $\Res^H_{H_s} W$ are disjoint.
  \end{enumerate}
  \begin{proof}
    $V$ is irreducible if and only if $\inpd{V, V}_G = 1$.
    By Thm \ref{thm:frob}, we have
    \[ \inpd{V, V}_G = \inpd{\Ind^G_H W, V}_G = \inpd{W, \Res^G_H V}_H \]
    But from Thm \ref{thm:mackey} we have
    \[ \Res^G_H V = \bigoplus_{s\in H\bsl G/H} \Ind^H_{H_s} W_s. \]
    Applying Thm \ref{thm:frob} once more, we obtain:
    \[ \inpd{V, V}_G = \sum_{s\in H\bsl G/H} \inpd{W, \Ind^H_{H_s} W_s}_H
      = \sum_{s\in H\bsl G/H} \inpd{\Res^H_{H_s} W, W_s}_{H_s}
    \]
    We know that $\inpd{\Res^H_{H_s} W, W_s}_{H_s} \ge 0$ for any
    $s\in H\bsl G/H$.
    In particular, for $s = 1 \in H\bsl G/H$, we have $H_s = H$,
    $\inpd{\Res^H_{H_s} W, W_s}_{H_s} = \inpd{W, W}_H \ge 1$.
    So $\inpd{V, V}_G = 1$ if and only if
    $\inpd{W, W}_H = 1$ and $\inpd{\Res^H_{H_s} W, W_s}_{H_s} = 0$
    for all $s\ne 1$ in $H\bsl G/H$.
    Which are exactly the conditions (a) and (b).
  \end{proof}
\end{theorem}

\begin{coro}
  Let $H \lhd G$ and $W$ be a representation of $H$. $\Ind^G_H W$ is
  irreducible if and only if $W$ is irreducible and not isomorphic to
  any of its conjugates $W_s$ for $s \not\in H$.
  \begin{proof}
    We have $H_s = H$ for all $s$. So we get the result directly by
    applying the theorem above.
  \end{proof}
\end{coro}
%%%%%%%%%%%%%%%%%%%%%%%%%%%%%%%%%%%%%%%%%%%%%
% \bibliographystyle{plain}
% \bibliography{journal.bib}
% \begin{thebibliography}{99}
% \bibitem[1]{ex}\url{http://www.example.com/}
% \end{thebibliography}
\end{document}
